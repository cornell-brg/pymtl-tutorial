%=========================================================================
% code-tut3-sort-rtl
%=========================================================================

\begin{figure}

  \begin{lstlisting}[xleftmargin={0.9in}]
#=========================================================================
# SortUnitFlatRTL
#=========================================================================
# A register-transfer-level model explicitly represents state elements
# with s.tick concurrent blocks and uses s.combinational concurrent
# blocks to model how data transfers between state elements.

from pymtl import *

class SortUnitFlatRTL( Model ):

  #=======================================================================
  # Constructor
  #=======================================================================

  def __init__( s, nbits=8 ):

    #---------------------------------------------------------------------
    # Interface
    #---------------------------------------------------------------------

    s.in_val  = InPort (1)
    s.in_     = InPort [4](nbits)

    s.out_val = OutPort(1)
    s.out     = OutPort[4](nbits)

    #---------------------------------------------------------------------
    # Stage S0->S1 pipeline registers
    #---------------------------------------------------------------------

    s.val_S1 = Wire(1)
    s.elm_S1 = Wire[4](nbits)

    @s.tick_rtl
    def pipereg_S0S1():
      s.val_S1.next  = s.in_val if ~s.reset else 0
      for i in xrange(4):
        s.elm_S1[i].next = s.in_[i]

    #---------------------------------------------------------------------
    # Stage S1 combinational logic
    #---------------------------------------------------------------------

    s.elm_next_S1 = Wire[4](nbits)

    @s.combinational
    def stage_S1():

      # Sort elements 0 and 1

      if s.elm_S1[0] <= s.elm_S1[1]:
        s.elm_next_S1[0].value = s.elm_S1[0]
        s.elm_next_S1[1].value = s.elm_S1[1]
      else:
        s.elm_next_S1[0].value = s.elm_S1[1]
        s.elm_next_S1[1].value = s.elm_S1[0]

      # Sort elements 2 and 3

      if s.elm_S1[2] <= s.elm_S1[3]:
        s.elm_next_S1[2].value = s.elm_S1[2]
        s.elm_next_S1[3].value = s.elm_S1[3]
      else:
        s.elm_next_S1[2].value = s.elm_S1[3]
        s.elm_next_S1[3].value = s.elm_S1[2]

    ...
\end{lstlisting}

  \caption{\textbf{Interface and First Stage of Sort Unit RTL Model --}
    RTL model of four-element sort unit corresponding to
    Figure~\ref{fig-tut3-sort-rtl}.}
  \label{code-tut3-sort-rtl}

\end{figure}

