%=========================================================================
% code-tut3-sort-sim
%=========================================================================

\begin{figure}

  \begin{lstlisting}[xleftmargin={0.9in},gobble=2]
  opts = parse_cmdline()

  # Create input datasets

  ninputs = 100
  inputs  = []

  if opts.input == "random":
    for i in xrange(ninputs):
      inputs.append( [ randint(0,0xff) for i in xrange(4) ] )

  ...

  # Instantiate and elaborate the design

  model_impl_dict = {
    'cl'         : SortUnitCL,
    'rtl-flat'   : SortUnitFlatRTL,
    'rtl-struct' : SortUnitStructRTL,
  }

  model = model_impl_dict[ opts.impl ]()

  dump_vcd = ""
  if opts.dump_vcd:
    dump_vcd = "sort-" + opts.impl + "-" + opts.input + ".vcd"

  model.vcd_file = dump_vcd

  model.elaborate()
  sim = SimulationTool( model )
  sim.reset()

  # Tick simulator until evaluation is finished

  counter = 0
  while counter < ninputs:

    if model.out_val:
      counter += 1

    if inputs:
      model.in_val.value = True
      for i,v in enumerate( inputs.pop() ):
        model.in_[i].value = v

    else:
      model.in_val.value = False
      for i in xrange(4):
        model.in_[i].value = 0

    if opts.trace:
      sim.print_line_trace()

    sim.cycle()

  # Report various statistics

  if opts.stats:
    print( "num_cycles          = {}".format( sim.ncycles ) )
    print( "num_cycles_per_sort = {:1.2f}".format( sim.ncycles/(1.0*ninputs) ) )
\end{lstlisting}

  \caption{\textbf{Simplified Simulator Script for Sort Unit --} The
    simulator script is responsible for handling command line arguments,
    creating input datasets, instantiating and elaborating the design,
    ticking the simulator until the evaluation is finished, and reporting
    various statistics.}
  \label{code-tut3-sort-sim}

\end{figure}

