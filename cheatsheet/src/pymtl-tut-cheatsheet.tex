%=========================================================================
% ECE 4750 Cheat Sheet
%=========================================================================

\documentclass{cbxdoc}

\usepackage{lscape}
\usepackage{listings}

\definecolor{cbxredC}  {RGB}{176,  24,  24}
\definecolor{cbxgreenC}{RGB}{ 88, 171,  30}
\definecolor{cbxblueC} {RGB}{ 69, 136, 214}
\definecolor{cbxbrownC}{RGB}{143, 132,  19}
\input{cbxpymtltut}

\geometry{margin=0.2in,hmargin=0.35in,tmargin=0.4in}

%-------------------------------------------------------------------------
% Document
%-------------------------------------------------------------------------

\begin{document}

\pagestyle{empty}

\begin{landscape}
\small

\begin{center}
  \textbf{\Large PyMTL
    Tutorial  \hspace{0.5em}\textbullet\hspace{0.5em}
    Princeton 2018 \hspace{0.5em}\textbullet\hspace{0.5em}
    Cheat Sheet}
\end{center}

\begin{minipage}[t]{3.25in}
\vspace{0pt}

\colorbox{gray!30!white}{\parbox{1.025\tw}{\rule[-0.4em]{0pt}{1.4em}\centering\textbf{%
  Tutorial Setup%
}}}

\smallskip\smallskip
\begin{verbatim}
  Username: tutorial
  Password: pymtl

  % cd ~/pymtl-tut
\end{verbatim}

\vspace{0.15in}
\colorbox{gray!30!white}{\parbox{1.025\tw}{\rule[-0.4em]{0pt}{1.4em}\centering\textbf{%
  Python Basics%
}}}

\begin{center}
Whitespace matters! Your code will not run correctly \\ if you use
improper indentation.
\end{center}

\vspace{-0.1in}
\begin{lstlisting}
# this is a comment

# Assignment to variables

a = 3
b = 5
c = a + b

# Printing values

print "hello world!"
print 3 + 5

# If conditionals

if test:
  # do stuff if test is true
elif test2:
  # do stuff if test2 is true
else:
  # do stuff if both tests are false

# For loops

for x in a_sequence:
  # do stuff for each member of a_sequence
  # for example, each item in a list

for x in range(10):
  # do stuff 10 times (0 through 9)

# Lists

lst = [ 1, 2, 3, 4 ] # create list with 4 elements
print lst[0]         # print first element
print lst[0:2]       # print first two elements

for x in lst:        # iterate over list and ...
  print x            #  ... print each element
\end{lstlisting}

\end{minipage}%
\hfill%
\begin{minipage}[t]{3.25in}
\vspace{0pt}

\colorbox{gray!30!white}{\parbox{1.025\tw}{\rule[-0.4em]{0pt}{1.4em}\centering\textbf{%
  PyMTL Model Template%
}}}

\smallskip
\begin{lstlisting}
class MyModel( Model ):
  def __init__( s, dtype ):

    # port declarations

    s.in_ = InPort( dtype )
    s.out = [OutPort( dtype ) for _ in range( 4 )]

    # submodule declaration and connections

    s.mod = SubMod( dtype, other, params )

    s.connect( s.in_, s.mod.in_ )
    s.connect_pairs(
      s.mod.out[0], s.out[0],
      s.mod.out[1], s.out[1],
    )

    # concurrent logic specifications

    @s.combinational
    def comb_logic():
      s.out[2].value = s.in_

    @s.tick_rtl
    def seq_logic():
      s.out[3].next = s.in_

    # wire declarations

    s.this_out = Wire( dtype )
    s.last_out = Wire( dtype )

    # more concurrent logic specifications

    @s.combinational
    def comb_logic2():
      s.this_out.value = s.in_

    @s.tick_rtl
    def seq_logic2():
      if s.reset: s.last_out.next = 0
      else:       s.last_out.next = s.this_out
\end{lstlisting}

\end{minipage}%
\hfill%
\begin{minipage}[t]{3.25in}
\vspace{0pt}

\colorbox{gray!30!white}{\parbox{1.025\tw}{\rule[-0.4em]{0pt}{1.4em}\centering\textbf{%
  PyMTL Modeling Guidelines%
}}}

\smallskip\smallskip\raggedright
\begin{cbxlist}{1.5em}{0em}{1em}

  \item Data is communicated between PyMTL models using \TT{InPorts} \&
     \TT{OutPorts}, not method calls! Models are composed
     \BF{structurally} by connecting their ports using \TT{s.connect}
     statements.

  \item Instantiated \TT{InPort}, \TT{OutPort}, and \TT{Wire} objects must
         stored as members of the model class:

\begin{verbatim}
  s.signal_name = InPort(8)
\end{verbatim}

  \item Instantiated submodels must also be stored as members of the
     model class:

\begin{verbatim}
  s.model_name = Incr( dtype=8, incr_amt=1 )
\end{verbatim}

  \item \BF{Sequential} concurrent logic blocks must always be annotated
     with \TT{@s.tick\_(fl|cl|rtl)}. \\
     \TT{InPorts}/\TT{OutPorts}/\TT{Wires} updated in \BF{sequential}
     blocks must write the \TT{.next} attribute.

\begin{verbatim}
  @s.tick_rtl
  def sequential_block():
    s.out.next = s.a & s.b
\end{verbatim}

  \item \BF{Combinational} concurrent logic blocks must always be
     annotated with \TT{@s.combinational}.
     \TT{InPorts}/\TT{OutPorts}/\TT{Wires} updated in \BF{combinational}
     blocks must write the \TT{.value} attribute.

\begin{verbatim}
  @s.combinational
  def sequential_block():
    s.out.value = s.a & s.b
\end{verbatim}

\end{cbxlist}

\end{minipage}%

\end{landscape}

\begin{landscape}
\small

\begin{center}
  \textbf{\Large PyMTL
    Tutorial  \hspace{0.5em}\textbullet\hspace{0.5em}
    Princeton 2018 \hspace{0.5em}\textbullet\hspace{0.5em}
    Cheat Sheet}
\end{center}

\hfill
\begin{minipage}[t]{3.25in}
\vspace{0pt}

\colorbox{gray!30!white}{\parbox{1.025\tw}{\rule[-0.4em]{0pt}{1.4em}\centering\textbf{%
  Using the PyMTL SimulationTool%
}}}

\smallskip
\begin{lstlisting}[numbers={none},xleftmargin={0.1in}]
model = MyModel( dtype = 8 )
model.elaborate()
sim = SimulationTool()

for i in range( 10 ):
  model.in_.value = i
  sim.cycle()
  sim.print_line_trace()
\end{lstlisting}

\vspace{0.15in}
\colorbox{gray!30!white}{\parbox{1.025\tw}{\rule[-0.4em]{0pt}{1.4em}\centering\textbf{%
  Using the PyMTL TranslationTool%
}}}

\smallskip
\begin{lstlisting}[numbers={none},xleftmargin={0.1in}]
model = MyModel( dtype = 8 )
model.elaborate()
model = TranslationTool( model )
\end{lstlisting}

\vspace{0.15in}
\colorbox{gray!30!white}{\parbox{1.025\tw}{\rule[-0.4em]{0pt}{1.4em}\centering\textbf{%
  Writing PyMTL Unit Tests with py.test%
}}}

\smallskip
\begin{lstlisting}[numbers={none},xleftmargin={0.1in}]
@pytest.mark.parameterize('dtype', 8, 32 )
def test_my_model( test_verilog, dump_vcd, dtype ):

   model = MyModel( dtype )
   model.elaborate()
   model.vcd_file = dump_vcd

   if test_verilog:
     model = TranslationTool( model )

   sim = SimulationTool( model )

   print()
   for i in range( 10 ):
     data = random.randrange( 2**dtype )
     model.in_.value = data
     sim.cycle()
     sim.print_line_trace()
     assert model.out[2].value = data
\end{lstlisting}

\vspace{0.15in}
\colorbox{gray!30!white}{\parbox{1.025\tw}{\rule[-0.4em]{0pt}{1.4em}\centering\textbf{%
  Running PyMTL Unit Tests with py.test%
}}}

\vspace{0.1in}
\begin{verbatim}
 % py.test ../my_model_dir --verbose --tb=line
 % py.test ../my_model_dir -k <grep_str>
 % py.test ../my_model_dir -k <grep_str> -s

 % py.test ../my_model_dir --test-verilog
 % py.test ../my_model_dir --test-verilog --dump-vcd
\end{verbatim}

\end{minipage}%
\hfill%
\begin{minipage}[t]{3.25in}
\vspace{0pt}

\colorbox{gray!30!white}{\parbox{1.025\tw}{\rule[-0.4em]{0pt}{1.4em}\centering\textbf{%
  PyMTL Bits Operators
}}}

\newenvironment{optbl}[1]
{
  \vspace{0.5em}
  \hspace{1em}\BF{#1}\vphantom{g}
  \smallskip

  \hspace{0.75em}
  \begin{tabular}{>{\ttfamily\centering\arraybackslash}p{0.65in}p{2.2in}}
\toprule
}{
  \end{tabular}
}

\vspace{0.1in}
\begin{optbl}{Logical Operators}
  \verb|&|   & bitwise AND   \\
  |          & bitwise OR    \\
  \verb|^|   & bitwise XOR   \\
  \verb|^~|  & bitwise XNOR  \\
  \verb|~|   & bitwise NOT   \\
\end{optbl}

\smallskip

\begin{optbl}{Arithmetic Operators}
  \verb|+|   & addition         \\
  \verb|-|   & subtraction      \\
\end{optbl}

\smallskip

\begin{optbl}{Reduction Operators}
  \verb|reduce_and| & reduce via AND \\
  \verb|reduce_or|  & reduce via OR  \\
  \verb|reduce_xor| & reduce via XOR \\
\end{optbl}

\smallskip

\begin{optbl}{Shift Operators}
  \verb|>>|  & shift right            \\
  \verb|<<|  & shift left             \\
\end{optbl}

\smallskip

\begin{optbl}{Slice Operators}
  \verb|[x]|   & get/set bit x            \\
  \verb|[x:y]| & get/set bits x upto y    \\
\end{optbl}

\smallskip

\begin{optbl}{Relational Operators}
  \verb|==|  & equal                  \\
  \verb|!=|  & not equal              \\
  \verb|>|   & greater than           \\
  \verb|>=|  & greater than or equals \\
  \verb|<|   & less than              \\
  \verb|<=|  & less than or equals    \\
\end{optbl}

\smallskip

\begin{optbl}{Other Functions}
  \verb|concat| & concatenate         \\
  \verb|sext|   & sign-extension      \\
  \verb|zext|   & zero-extension      \\
\end{optbl}

\end{minipage}
\hfill\mbox{}

\end{landscape}

\end{document}
