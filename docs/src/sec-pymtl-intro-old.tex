%=========================================================================
% Presentation: PyMTL Intro
%=========================================================================

\section[Presentation: PyMTL Intro]{Presentation: PyMTL Introduction}

\begin{frame}{Computer Architecture Research Methodologies}
  \cbxfigc<1\h0>{pymtl-stack_0-split.svg.pdf}
  \cbxfigc<2\h0>{pymtl-stack_0_1_1a-split.svg.pdf}
  \cbxfigc<3\h0>{pymtl-stack_0_1_2_2a-split.svg.pdf}
  \cbxfigc<4\h0>{pymtl-stack_0_1_2_3_3a-split.svg.pdf}
  \cbxfigc<5\h1>{pymtl-stack_0_1_2_3_4-split.svg.pdf}
  \cbxfigc<6\h0>{pymtl-stack_0_4_5-split.svg.pdf}
  \cbxfigc<7\h2>{pymtl-stack_4_5_6-split.svg.pdf}
\end{frame}

\begin{frame}{Great Ideas From Prior Frameworks}
  \insertslide{pymtl-intro}{8}
\end{frame}

\begin{frame}{What is PyMTL?}
  \insertslides{pymtl-intro}{9}{13}
\end{frame}

\begin{frame}{The PyMTL Framework}
  \insertslides{pymtl-intro}{14}{19}
\end{frame}

\begin{frame}{What is PyMTL for and not (currently) for?}
\begin{cbxlist}

  \1 \BF{PyMTL is for ...}

     \2 Taking an accelerator design from concept to implementation
     \2 Construction of highly-parameterizable RTL chip generators
     \2 Rapid design, testing, and exploration of hardware mechanisms
     \2 Quickly prototyping models and interfacing them with GEM5
     \2 Interfacing models with imported Verilog

  \1 \BF{PyMTL is not (currently) for ...}

     \2 Python high-level synthesis
     \2 Many-core simulations with hundreds of cores
     \2 Full-system simulation with real OS support
     \2 Users needing a complex OOO processor model ``out of the box''
     \2 Users needing an ARM/x86 processor model ``out of the box''
     \2 Users needing a mature modeling framework that will not change

\end{cbxlist}
\end{frame}

\begin{frame}{Why Python?}

  \cbxfloatright{\cbxfigc[0.21\tw]{python-logo.svg.pdf}}

  \begin{cbxlist}

    \1 Python is well regarded as a highly productive \\ language with
       lightweight, pseudocode-like syntax

    \1 Python supports modern language features to \\ enable rapid, agile
       development (dynamic typing, \\ reflection, metaprogramming)

    \1 Python has a large and active developer and support community

    \1 Python includes extensive standard and third-party libraries

    \1 Python enables embedded domain-specific languages

    \1 Python facilitates engaging application-level researchers

    \1 Python includes built-in support for integrating with C/C++

    \1 Python performance is improving with advanced JIT compilation

  \end{cbxlist}

\end{frame}

\begin{frame}{PyMTL 101: Traditional Model in Python}
  \insertslide{pymtl-intro}{20}
\end{frame}

\begin{frame}{PyMTL 101: Model in PyMTL Embedded DSL}
  \insertslides{pymtl-intro}{21}{25}
\end{frame}

