%=========================================================================
% Presentation: PyMTL Intro
%=========================================================================

\section[Presentation: PyMTL Intro]{Presentation: PyMTL Introduction}

\begin{frame}{Multi-Level Modeling Methodologies}
  \cbxfigc<1\h0>{pymtl-stack-v2_0-split.svg.pdf}
  \cbxfigc<2\h0>{pymtl-stack-v2_0_2_2a-split.svg.pdf}
  \cbxfigc<3\h0>{pymtl-stack-v2_0_1_1a_2-split.svg.pdf}
  \cbxfigc<4\h0>{pymtl-stack-v2_0_1_2_3_3a-split.svg.pdf}
  \cbxfigc<5\h1>{pymtl-stack-v2_0_1_2_3_4-split.svg.pdf}
  \cbxfigc<6\h0>{pymtl-stack-v2_0_4_5-split.svg.pdf}
  \cbxfigc<7\h0>{pymtl-stack-v2_4_5_6-split.svg.pdf}
  \cbxfigc<8\h0>{pymtl-stack-v2_4_5_6_6a-split.svg.pdf}
  \cbxfigc<9\h2>{pymtl-stack-v2_4_5_6_6a_6b-split.svg.pdf}
\end{frame}

\begin{frame}{VLSI Design Methodologies}
  \cbxfigc<1\h0>{vlsi-methodology_0-split.svg.pdf}
  \cbxfigc<2\h0>{vlsi-methodology_0_1-split.svg.pdf}
  \cbxfigc<3\h1>{vlsi-methodology_0_1_2-split.svg.pdf}
\end{frame}

\begin{frame}{\IT{Productive} Multi-Level Modeling \IT{and} VLSI Design}
  \cbxfigc<1\h0>{hgsf_0-split.svg.pdf}
  \cbxfigc<2\h0>{hgsf_0_1-split.svg.pdf}
  \cbxfigc<3\h1>{hgsf_0_1_2-split.svg.pdf}
\end{frame}


\begin{frame}[c]

  \cbxfigc[0.5\tw]{pymtl-flat.svg.pdf}

\medskip
\begin{cbxblock}
  \Large PyMTL is a Python-based hardware generation \\ and simulation
  framework for SoC design \\ which enables productive \\ multi-level modeling and VLSI implementation
\end{cbxblock}
\end{frame}

\begin{frame}{The PyMTL Framework}
  \cbxfigc<1\h0>{pymtl-flow_0-split.svg.pdf}
  \cbxfigc<2\h0>{pymtl-flow_0_1-split.svg.pdf}
  \cbxfigc<3\h0>{pymtl-flow_0_1_2-split.svg.pdf}
  \cbxfigc<4\h0>{pymtl-flow_0_1_2_3-split.svg.pdf}
  \cbxfigc<5-\h1>{pymtl-flow_0_1_2_3_4-split.svg.pdf}

  % \begin{onlyenv}<6\h2>
  % \begin{tikzpicture}[remember picture,overlay]
  %   \draw[draw=none,fill=white,opacity=0.70]
  %   (-1,0)
  %   rectangle
  %   (12,8.25);
  %   \node[fill=white,text=red,font=\Huge] at (current page.center)
  %   {But isn't Python too slow?};
  % \end{tikzpicture}
  % \end{onlyenv}

\end{frame}

\begin{frame}[fragile]{PyMTL v2 Syntax and Semantics}

\vspace{-0.2in}
\begin{lstlisting}
from pymtl import *

class RegIncrRTL( Model ):

  def __init__( s, dtype ):
    s.in_  = InPort ( dtype )
    s.out  = OutPort( dtype )
    s.tmp  = Wire   ( dtype )

    @s.tick_rtl
    def seq_logic():
      s.tmp.next = s.in_

    @s.combinational
    def comb_logic():
      s.out.value = s.tmp + 1
\end{lstlisting}

\vspace{-2.5in}\hfill
\cbxfigc[0.4\tw]{tut3-regincr.svg.pdf}
\end{frame}

\begin{frame}[fragile]

\vspace{0.3in}
\begin{lstlisting}[basicstyle={\ttfamily\footnotesize}]
class GcdUnitFL( Model ):
 def __init__( s ):

  # Interface
  s.req    = InValRdyBundle  ( GcdUnitReqMsg() )
  s.resp   = OutValRdyBundle ( Bits(16)        )

  # Adapters (e.g., TLM Transactors)
  s.req_q  = InValRdyQueueAdapter  ( s.req  )
  s.resp_q = OutValRdyQueueAdapter ( s.resp )

  # Concurrent block
  @s.tick_fl
  def block():
   req_msg = s.req_q.popleft()
   result = gcd( req_msg.a, req_msg.b )
   s.resp_q.append( result )
\end{lstlisting}

\vspace{-3.3in}\hfill
\cbxfigc[0.6\tw]{tut3-gcd-fl.svg.pdf}

\begin{onlyenv}<2\h0>
\vspace{2in}
\begin{tikzpicture}[remember picture,overlay]
  \draw[draw=none,fill=white,opacity=0.70]
  (-1,0)
  rectangle
  (12,8);
  \node[fill=white,text=red,font=\Huge] at (current page.center)
  {But isn't Python too slow?};
\end{tikzpicture}
\end{onlyenv}

\end{frame}

% \begin{frame}{What Does PyMTL Enable?}
%   %\pymtlslides{29}{36}
%   \pymtlslide<1\h1>{31}
%   \pymtlslide<2\h2>{33}
%   \pymtlslide<3\h3>{34}
%   \pymtlslide<4\h4>{35}
% \end{frame}
%
% \begin{frame}{What Does PyMTL Enable?}
%   \pymtlslide{37}
%
%   \vspace{-1.55in}
%   \colorbox{white}{
%     \begin{minipage}[c][0.25in][t]{\tw}\small
%       \vspace{0.05in}\hspace{0.7em}
%       (Used to implement CL model in [MICRO'14,MICRO'18])
%     \end{minipage}
%   }
%
% \begin{onlyenv}<2\h2>
% \vspace{1.45in}
% \begin{tikzpicture}[remember picture,overlay]
%   \draw[draw=none,fill=white,opacity=0.70]
%   (0,0)
%   rectangle
%   (12,7);
%   \node[fill=white,text=red,font=\Huge] at (current page.center)
%   {But isn't Python too slow?};
% \end{tikzpicture}
% \end{onlyenv}
%
% \end{frame}

\begin{frame}{Performance/Productivity Gap}
Python is growing in popularity in many domains of scientific and
high-performance computing. \RD{How do they close this gap?}

\medskip\hspace{0.15in}
\begin{cbxlist}
  \1 Python-Wrapped C/C++ Libraries \\
     (NumPy, CVXOPT, NLPy, pythonoCC, gem5)
  \1 Numerical Just-In-Time Compilers \\
     (Numba, Parakeet)
  \1 Just-In-Time Compiled Interpreters \\
     (PyPy, Pyston)
  \1 Selective Embedded Just-In-Time Specialization \\
     (SEJITS)
\end{cbxlist}

\end{frame}

% \begin{frame}{PyMTL Hybrid Python/C++ Co-Simulation}
%   \pymtlslides{77}{82}
%   %\pymtlslide{82}
% \end{frame}

% \begin{frame}{PyMTL Results: 64-Node Mesh Network}
%   \cbxfigc<1\h0>{pymtl-results_0_0a-split.svg.pdf}
%   \cbxfigc<2\h0>{pymtl-results_0_1_1a-split.svg.pdf}
%   \cbxfigc<3\h0>{pymtl-results_0_1_2_2a-split.svg.pdf}
%   \cbxfigc<4\h1>{pymtl-results_0_1_2_3_3a-split.svg.pdf}
%
%   \vspace{-0.1in}
%   \begin{center}\small
%   RTL model of 64-node mesh network with single-cycle routers, elastic
%   buffer flow control, uniform random traffic, with an injection rate
%   just before saturation
%   \end{center}
%
% \end{frame}

\begin{frame}{Evaluating HDLs, HGFs, and HGSFs}
\begin{cbxlist}
  \1 Apple-to-apple comparison of simulator performance
  \1 64-bit radix-four integer iterative divider
  \1 All implementations use same control/datapath split \\ with the same
     level of detail
  \1 Modeling and simulation frameworks:
    \2 Verilog: Commercial verilog simulator, Icarus, Verilator
    \2 HGF: Chisel
    \2 HGSFs: PyMTL, MyHDL, PyRTL, Migen
\end{cbxlist}
\end{frame}

\begin{frame}{Productivity/Performance Gap}

\cbxfigc<1\h1>{mamba-results-3.png}
\begin{onlyenv}<1\h1>
\vspace{-2in}\hfill
\colorbox{white}{%
  \begin{minipage}[c][2in][t]{0.8\tw}
    \mbox{\ }
  \end{minipage}
}

\vspace{-1.75in}\hspace*{1.25in}
\begin{cbxlist}
  \1 Higher is better
  \1 Log scale (gap is larger than it seems)
  \1 Commercial Verilog simulator is \\ 20\X{} faster than Icarus
 \1 Verilator requires C++ testbench, \\ only works with synthesizable
  code, \\ takes significant time to compile, \\ but is 200\X{} faster than
  Icarus
\end{cbxlist}
\end{onlyenv}

\cbxfigc<2\h2>{mamba-results-3.png}
\begin{onlyenv}<2\h2>
\vspace{-2in}\hfill
\colorbox{white}{%
  \begin{minipage}[c][2in][t]{0.67\tw}
    \mbox{\ }
  \end{minipage}
}

\medskip
\begin{cbxlist}[t]
  \1 Chisel (HGF) generates Verilog and uses Verilog simulator
\end{cbxlist}
\end{onlyenv}

\cbxfigc<3\h3>{mamba-results-3.png}
\begin{onlyenv}<3\h3>
\vspace{-1.9in}\hfill
\colorbox{white}{%
  \begin{minipage}[c][1.9in][t]{0.088\tw}
    \mbox{\ }
  \end{minipage}
}

\medskip
\begin{cbxlist}[t]
  \1 Using CPython interpreter, Python-based HGSFs are much slower than
  commercial Verilog simulators; even slower than Icarus!
\end{cbxlist}
\end{onlyenv}

\cbxfigc<4\h4>{mamba-results-4.png}
\begin{onlyenv}<4\h4>
\vspace{-1.9in}\hfill
\colorbox{white}{%
  \begin{minipage}[c][1.9in][t]{0.088\tw}
    \mbox{\ }
  \end{minipage}
}

\medskip
\begin{cbxlist}[t]
  \1 Using PyPy JIT compiler, Python-based HGSFs achieve $\approx$10\X{}
  speedup, but still significantly slower than commercial Verilog
  simulator
\end{cbxlist}
\end{onlyenv}

\cbxfigc<5-\h5->{mamba-results-5.png}
\begin{onlyenv}<5\h5>
\vspace{-1.9in}\hfill
\colorbox{white}{%
  \begin{minipage}[c][1.9in][t]{0.088\tw}
    \mbox{\ }
  \end{minipage}
}

\medskip
\begin{cbxlist}[t]
  \1 Hybrid C/C++ co-simulation improves performance but:
  \2 only works for a synthesizable subset
  \2 may require designer to simultaneously work with C/C++ and Python
\end{cbxlist}
\end{onlyenv}

\begin{onlyenv}<6\h6>
\medskip
\begin{cbxlist}
  \1 Mamba is \BF{20\X{}} faster than PyMTLv2, \BF{4.5\X{}} faster that
  PyMTLv2 with hybrid co-simulation, comparable to commercial
  simulators \1 Mamba uses careful co-optimization of the framework and
  the JIT
\end{cbxlist}
\end{onlyenv}

\end{frame}

% \begin{frame}[fragile]{PyMTL v3 Syntax and Semantics}
%
% \vspace{-0.2in}
% \begin{lstlisting}
% from pymtl import *
%
% class RegIncrRTL( Model ):
%
%   def __init__( s, dtype ):
%     s.in_  = InValuePort ( dtype )
%     s.out  = OutValuePort( dtype )
%     s.tmp  = Wire        ( dtype )
%
%     @s.update_on_edge
%     def seq_logic():
%       s.tmp = s.in_
%
%     @s.update
%     def comb_logic():
%       s.out = s.tmp + 1
% \end{lstlisting}
%
% \vspace{-2.5in}\hfill
% \cbxfigc[0.4\tw]{tut3-regincr.svg.pdf}
%
% \medskip
% \hfill\cbxfigc[0.5\tw]{update-block-schedule.svg.pdf}
%
% \end{frame}

% \begin{frame}{PyMTL v3 Performance}
%
% \hspace*{0.02in}
% \begin{tabular}{lc<{\onslide<2->}c<{\onslide<3->}c<{\onslide<4->}c<{\onslide}}
% \toprule
% \centering Technique  & Divider  & 1-Core   & 16-core  & 32-core\\
% \midrule
% Event-Driven          & 24K CPS  & 6.6K CPS & 155 CPS  & 66 CPS \\
% \midrule
% \multicolumn{5}{l}{\BF{JIT-Aware HGSF}}\\
% + Static Scheduling   & 13\X{}   & 2.6\X{}  & 1\X{}    & 1.1\X{} \\
% + Schedule Unrolling  & 16\X{}   & 24\X{}   & 0.4\X{}  & 0.2\X{} \\
% + Heuristic Toposort  & 18\X{}   & 26\X{}   & 0.5\X{}  & 0.3\X{} \\
% + Trace Breaking      & 19\X{}   & 34\X{}   & 2\X{}    & 1.5\X{} \\
% + Consolidation       & 27\X{}   & 34\X{}   & 47\X{}   & 42\X{}  \\
% \midrule
% \multicolumn{5}{l}{\BF{HGSF-Aware JIT}}\\
% + RPython Constructs   & 96\X{}   & 48\X{}   & 62\X{}   & 61\X{} \\
% + Huge Loop Support   & 96\X{}   & 49\X{}   & 65\X{}   & 67\X{} \\
% \bottomrule
% \end{tabular}
%
% \begin{onlyenv}<2->
% \smallskip
% \hspace*{0.1in}
% \begin{cbxlist}[t]
%   \1 RISC-V RV32IM five-stage pipelined cores
%   \1 Only models cores, no interconnect nor caches
% \end{cbxlist}
% \end{onlyenv}
% \end{frame}

% \begin{frame}{PyMTL v3 Performance with Overheads}
%   \cbxfigc[0.95\tw]{mamba-results-mcore-1.png}
%
%   \parbox{0.5\tw}{\centering\BF{Simulating 1 RISC-V Core}}%
%   \parbox{0.5\tw}{\centering\BF{Simulating 32 RISC-V Cores}}
%
%   \smallskip
%   \cbxfigc{mamba-perf-eq.png}
%
% \begin{onlyenv}<1\h0>
% \vspace{-2.7in}\hfill
% \colorbox{white}{%
%   \begin{minipage}[c][2.25in][t]{0.5\tw}
%     \mbox{\ }
%   \end{minipage}
% }
% \end{onlyenv}
% \begin{onlyenv}<2\h0>
%
% \end{onlyenv}
% \end{frame}

\begin{frame}{PyMTL ASIC Tapeouts}
\begin{cbxcols}

  \begin{column}{0.46\tw}
    \cbxfigbc[0.93\tw]{asic-chip-annotated.png}

    \small\centering

    \smallskip \BF{BRGTC1 in 2016} \\ RISC processor, 16KB SRAM \\
    HLS-generated accelerator

    \smallskip
    2x2mm, 1.2M-trans, IBM 130nm
  \end{column}

  \begin{column}{0.46\tw}
    \cbxfigc[0.935\tw]{brgtc2.svg.pdf}

    \small\centering

    \smallskip \BF{BRGTC2 in 2018} \\ 4xRV32IMAF cores with ``smart'' sharing
    L1\$/LLFU, PLL

    \smallskip
    1x1.2mm, 6.7M-trans, TSMC 28nm

  \end{column}

\end{cbxcols}
\end{frame}

\begin{frame}{PyMTL and Open-Source Hardware}
\begin{cbxlist}

 \1 State-of-the-art in open-source HDL simulators
    \2 \IT{Icarus Verilog:} Verilog interpreter-based simulator
    \2 \IT{Verilator:} Verilog AOT-compiled simulator
    \2 \IT{GHDL:} VHDL AOT-compiled simulator
    \2 No open-source simulator supports modern verification environments

 \1 PyMTL as an open-source design, simulation, verification environment

    \2 Open-source hardware developers can use Verilog RTL for design and
       Python, a well-known general-purpose language, for verification

    \smallskip
    \2 PyMTL for FL design enables creating high-level golden reference
       models

    \smallskip
    \2 PyMTL for RTL design enables creating highly parameterized
       hardware components which is critical for encouraging reuse in an
       open-source ecosystem

\end{cbxlist}
\end{frame}

\begin{frame}{PyMTL and Open-Source Hardware}
\cbxfigc{pymtl-teaching-posh.svg.pdf}
\end{frame}

