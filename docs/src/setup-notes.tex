
\documentclass[11pt]{article}

\usepackage[letterpaper, margin=1in]{geometry}

%\usepackage{libertine}
%\usepackage{inconsolata}
\usepackage{mathpazo}
%\usepackage{courier}
\usepackage[T1]{fontenc}

\setlength{\parindent}{0em}
\setlength{\parskip}{0.5em}

% Lists

\newenvironment{cbxlist}
{%
  \begin{list}{\textbullet}
  {%
    \setlength{\leftmargin}{1.5em}
    \setlength{\rightmargin}{0em}
    \setlength{\topsep}{0.05in}
    \setlength{\parsep}{0pt}
    \setlength{\listparindent}{0pt}
    \setlength{\itemsep}{0.5em}
  }
}{%
  \end{list}
}

\begin{document}
\pagestyle{empty}

\begin{center}

  {\LARGE\textbf{PyMTL Tutorial}}

  \vspace{0.05in}
  {\LARGE\textbf{Virtual Machine Installation and Set-Up Instructions}}

\end{center}

\vspace{0.1in}
Plug in the USB drive. If you do not have VirtualBox installed on your
computer, you need to install it. We have provided installers for Windows,
Mac OS X, and Linux computers. The installers are included in the directory
labeled \texttt{virtualbox}. Depending on your platform, locate the
appropriate installer:

\begin{cbxlist}

  \item If your operating system is Windows, double click
  \texttt{VirtualBox-6.0.8-130520-Win.exe}, located inside \texttt{windows}
  dircrtry.
  \item If your operating system is Mac OS X, double click
  \texttt{VirtualBox-6.0.8-130520-OSX.dmg}, located inside \texttt{osx}
  directory.
  \item If your operating system is Linux, double click
    \texttt{VirtualBox-6.0-6.0.8\_130520\_...} located inside
    the directory that better refelcts your distro. If your distro is not
    included, double click
    \texttt{VirtualBox-6.0.8-130520-Linux\_amd64.run} inside \texttt{generic}.
    If double-clicking does not work, you might need to run the file from
    the terminal:

    \begin{verbatim}
      % sudo dpkg -i {FILE_NAME}.deb    # For .deb files
      % sudo rpm  -i {FILE_NAME}.rpm    # For .rpm files
      % sudo sh      {FILE_NAME}.run    # For .run files
    \end{verbatim}

\end{cbxlist}
\vspace{-0.15in}

Follow the directions to install VirtualBox. You might need to enter your
administrator password.

Once VirtualBox is installed, double click the \texttt{pymtl.ova}
file, which will decompress and import the virtual machine. The default
configurations included in the \texttt{OVA} file are recommended to run
the virtual machine. You can, however, increase the memory to have a better
experience if your machine allows it. The virtualmachine will use
about 10 GB on your local hard drive. Once the import is complete, you
can start the virtual machine by clicking the ``Start'' button in
VirtualBox. You will see a window pop up and the virtual machine will boot.

Once fully booted, you will be automatically logged in. Then, you can
resize the window or make it full screen to make it easier to work with
on your laptop. If the font is too small, you can change the scaling factor
in the machine settings. To open the machine settings, click on
\texttt{Machine} -> \texttt{Settings}. Then, change the \texttt{Scale Factor}
in the \texttt{Display} tab.

Many activities require the use of command line. For these
activities, double click the \texttt{Terminal} icon on the desktop, and run
the commands as directed. The activities will also require you to edit files.
For this, we recommend that you use \texttt{geany} unless you have a
preference to use more advanced editors such as \texttt{vim} or
\texttt{emacs}. To familiarize yourself with editing on the virtual machine,
practice editing and saving a file in your home directory. Type the
following commands in the shell in the \texttt{Terminal} window. The
\texttt{\%} characters indicate the shell prompt, so do not type this
character when entering the commands:

\vspace{-0.05in}
\begin{verbatim}
  % geany ~/hello.txt
\end{verbatim}
\vspace{-0.05in}

Put some content in this file and save. Close the \texttt{geany} window.
Re-run the command to ensure your changes were saved:

\vspace{-0.05in}
\begin{verbatim}
  % geany ~/hello.txt
\end{verbatim}
\vspace{-0.05in}

\end{document}
