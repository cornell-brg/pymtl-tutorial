\documentclass[11pt]{article}

\usepackage[letterpaper, margin=1in]{geometry}

%\usepackage{libertine}
\usepackage{inconsolata}
\usepackage{mathpazo}
%\usepackage{courier}
\usepackage[T1]{fontenc}

\setlength{\parindent}{0em}
\setlength{\parskip}{0em}

\begin{document}
\pagestyle{empty}

\begin{center}
{\Large Virtual Machine Set-Up Instructions}
\end{center}

Plug in the USB drive. Virtual-machine related files are located under the
\texttt{vm/} directory. If you do not have VirtualBox installed on your
computer, you need to install it. We have provided installers for Windows,
Mac OS X, and Linux computers. Depending on your platform, double click on
the correct installer under the \texttt{vm/} directory:

\begin{itemize}
  \item If your operating system is Windows, double click
  \texttt{VirtualBox-4.3.28-100309-Win.exe}
  \item If your operating system is Mac OS X, double click
  \texttt{VirtualBox-4.3.28-100309-OSX.dmg}
  \item If your operating system is Linux, double click
    \texttt{VirtualBox-4.3.28-100309-Linux\_x86.run} or 
    \texttt{VirtualBox-4.3.28-100309-Linux\_amd64.run} depending if your OS is
    32-bit or 64-bit respectively. If this doesn't work, you might need to
    run the file from the terminal:
    \begin{verbatim}
      % sudo sh ./VirtualBox-4.3.28-100309-Linux\_x86.run
    \end{verbatim}
\end{itemize}

Follow the directions to install VirtualBox. You might need to enter your
administrator password.\\

Once VirtualBox is installed, double click \texttt{pymtl\_pydgin.ova}
file. This will decompress and import the virtual machine. This will use
about 10 GB on your local hard drive. Once the import is complete, you can
start the virtual machine.\\

You will see a window pop up and the virtual machine will boot. Once fully
booted, you will see the login screen. You have two options on how to
interact with the virtual machine: directly through the virtual machine
desktop or by local SSH. Unless you are an advanced Linux user, we
recommend that you use the virtual machine desktop.\\

%-------------------------------------------------------------------------
{\large \textbf{Option 1: Interacting through desktop}}\\
%-------------------------------------------------------------------------

Log in to the virtual machine desktop using the following credentials:

\begin{verbatim}
  User name: tutorial
  Password:  pymtl
\end{verbatim}

Once logged in, you can resize the window or make it full screen to make
it easier to work with on your laptop. Many activities require the use of
command line. For these activities, double click the \texttt{Terminal}
icon on the desktop, and run the commands as directed. The activities will
also require you to edit files. For this, we recommend that you use
\texttt{gedit} unless you have a preference to use more advanced editors
such as \texttt{vim} or \texttt{emacs}. To familiarize yourself with
editing on the virtual machine, practice editing and saving a file on your
home directory. Type the following commands in the shell in the
\texttt{Terminal} window. The \texttt{\%} characters indicate the shell
prompt, so do not type this character when entering the commands:

\begin{verbatim}
  % gedit ~/hello.txt
\end{verbatim}

Put some content in this file and save. Close the \texttt{gedit} window.
Re-run the command to ensure your changes were saved:

\begin{verbatim}
  % gedit ~/hello.txt
\end{verbatim}

%--------------------------------------------------------------------------
{\large \textbf{Option 2: Interacting using local SSH}}\\
%-------------------------------------------------------------------------

The virtual machine is set to listen to local port 22422 for SSH. You can
locally SSH into the virtual machine using a terminal on Mac or Linux, or
using an SSH client such as \texttt{PuTTY} on Windows:

\begin{verbatim}
  % ssh -p 22422 -Y tutorial@localhost
  User name: tutorial
  Password:  pymtl
\end{verbatim}

\texttt{gedit}, \texttt{vim}, and \texttt{emacs} are installed on the
virtual machines. If you need some other software, and if you have
internet access you can install it.  The tutorial account has
\texttt{sudo} access, so you can install using \texttt{yum} package
manager:

\begin{verbatim}
  % yum search <program>
  % sudo yum install <package name>
\end{verbatim}

If you want to transfer files between your local computer and the virtual
machine (for example to copy configuration files for editors), you can use
\texttt{scp} on your local machine by specifying the port (notice the
capital \texttt{-P} flag):

\begin{verbatim}
  % scp -P 22422 <local files> tutorial@localhost:<remote dir>
\end{verbatim}

One activity requires the use of \texttt{gtkwave} waveform viewer. The
\texttt{-Y} flag for the \texttt{ssh} command enables forwarding for X
Window System. X-forwarding might not work if you have a Windows or Mac
laptop. You should test out if you can see \texttt{gtkwave} window show
up:

\begin{verbatim}
  % gtkwave
\end{verbatim}

If \texttt{gtkwave} window does not show up, then you will need to use the
Option 1 for the activity that requires \texttt{gtkwave}.


\end{document}


