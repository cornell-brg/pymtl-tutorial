%=========================================================================
% code-tut3-basics-bitstruct
%=========================================================================

%\begin{figure}

  \begin{lstlisting}[gobble=4]
    # Point BitStruct

    >>> class Point( BitStructDefinition ):
    ...   def __init__( s ):
    ...     s.x = BitField(4)
    ...     s.y = BitField(4)
    ...
    >>> pt1 = Point()
    >>> pt1.x = 3
    >>> pt1.y = 4
    >>> pt1
    Bits( 8, 0x34 )
    >>> pt1.x
    Bits( 4, 0x3 )
    >>> pt1.y
    Bits( 4, 0x4 )

    >> pt1 & Bits( 8, 0xf0 )
    Bits( 8, 0x30 )
    >> pt1[0:4]
    Bits( 4, 0x3 )

    # Parameterized Point BitStruct

    >>> class PointN( BitStructDefinition ):
    ...   def __init__( s, nbits ):
    ...     s.x = BitField(nbits)
    ...     s.y = BitField(nbits)
    ...
    >>> Point8 = PointN(8)
    >>> pt2 = Point8()
    >>> pt2.x = 3
    >>> pt2.y = 4
    >>> pt2
    Bits( 16, 0x0304 )
    >>> pt2.x
    Bits( 8, 0x03 )
    >>> pt2.y
    Bits( 8, 0x04 )
  \end{lstlisting}

  \captionsetup{justification=centering}
  \captionof{figure}{\textbf{Creating and Using \TT{BitStruct} Objects}}
  \label{code-tut3-basics-bitstruct}

%\end{figure}
