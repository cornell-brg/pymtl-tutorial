%=========================================================================
% code-tut3-gcd-dpath
%=========================================================================

\begin{figure}

  \begin{lstlisting}[xleftmargin={0.9in}]
#=========================================================================
# GCD Unit RTL Datapath
#=========================================================================

class GcdUnitDpathRTL (Model):

  # Constructor

  def __init__( s ):

    #---------------------------------------------------------------------
    # Interface
    #---------------------------------------------------------------------

    s.req_msg_a  = InPort  (16)
    s.req_msg_b  = InPort  (16)
    s.resp_msg   = OutPort (16)

    # Control signals (ctrl -> dpath)

    s.a_mux_sel = InPort  (A_MUX_SEL_NBITS)
    s.a_reg_en  = InPort  (1)
    s.b_mux_sel = InPort  (B_MUX_SEL_NBITS)
    s.b_reg_en  = InPort  (1)

    # Status signals (dpath -> ctrl)

    s.is_b_zero = OutPort (1)
    s.is_a_lt_b = OutPort (1)

    #---------------------------------------------------------------------
    # Structural composition
    #---------------------------------------------------------------------

    # A mux

    s.sub_out   = Wire(16)
    s.b_reg_out = Wire(16)

    s.a_mux = m = Mux( 16, 3 )
    s.connect_dict({
      m.sel                  : s.a_mux_sel,
      m.in_[ A_MUX_SEL_IN  ] : s.req_msg_a,
      m.in_[ A_MUX_SEL_SUB ] : s.sub_out,
      m.in_[ A_MUX_SEL_B   ] : s.b_reg_out,
    })

    # A register

    s.a_reg = m = RegEn(16)
    s.connect_dict({
      m.en    : s.a_reg_en,
      m.in_   : s.a_mux.out,
    })
\end{lstlisting}

  \caption{\textbf{Excerpt from Datapath in GCD Unit RTL Model --} We use
    top-level constants for various control signal encodings (e.g.,
    \TT{A_MUX_SEL_NBITS}, \TT{A_MUX_SEL_IN}), and we use
    \TT{s.connect_dict} to enable more succinct structural composition in
    datapaths.}
  \label{code-tut3-gcd-dpath}

\end{figure}

